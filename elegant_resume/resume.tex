% This template is designed to offer an aesthetically pleasing resume that adheres to a formal and institutional tone, making it suitable for applications to companies and research centers requiring a high degree of professionalism. Navy blue has been chosen as the primary color to align with these objectives.
% The code is well-documented and annotated, allowing users to easily customize and modify it according to their needs. Please note that the template's content is meant to be humorous and should not be taken literally. We are grateful for your interest in using this template for your professional endeavors.
% Author: Christian Maria Giannetti

%----------------------------------------------------------------------------------------
%  Packages And Other Document Configurations
%----------------------------------------------------------------------------------------

\documentclass{resume} % Use the custom resume.cls style

% Document margins
\usepackage[left=0.75in,top=0.6in,right=0.75in,bottom=0.6in]{geometry}

% Color and hyperlink packages
\usepackage{xcolor}
\usepackage{hyperref}

% Footnote and margin adjustment packages
\usepackage{footnote}
\usepackage{changepage}

% Fontawesome package for icons
\usepackage{fontawesome}

% Tabularx package for custom tables
\usepackage{tabularx}

\usepackage{hyperref}

\hypersetup{
	colorlinks = true,
	breaklinks = true,
	allcolors = blue
}

% Define navyblue color
\definecolor{navyblue}{RGB}{0,54,123}

%----------------------------------------------------------------------------------------
%   Customizations
%----------------------------------------------------------------------------------------

% Define italicitem, bolditem, and plainitem commands
\newcommand{\italicitem}[1]{\item{\textit{#1}}}
\newcommand{\bolditem}[1]{\item{\textbf{#1}}}
\newcommand{\plainitem}[1]{\item{#1}}

% Define user-friendly link command for hyperlinks
\newcommand{\link}[2]{{\href{#1}{#2}}}

\newcommand{\entry}[2]{#1 & #2 \tabularnewline} % Defines an entry with two arguments: #1 for the first column and #2 for the second column

%----------------------------------------------------------------------------------------
%   Define envsection command for defining a new environment section
%----------------------------------------------------------------------------------------

\newcommand{\tableEnv}[2]{%
  \begin{rSection}{#1} % Begin rSection with the given name
    \begin{adjustwidth}{0.0in}{0.1in} % Set the left and right margins
      \begin{tabularx}{\linewidth}{@{} >{\bfseries}l @{\hspace{6ex}} X @{}}
        #2 % Print the content inside the tabularx environment
      \end{tabularx}
    \end{adjustwidth}
  \end{rSection}
}

%----------------------------------------------------------------------------------------
%   Begin document
%----------------------------------------------------------------------------------------

% Set name with navyblue color
\name{\color{navyblue} Boyuan Yu}

\begin{document}

%\printPersonalInfo{
%  \personalInfo{\tag{\faEnvelopeO}\info{pleasedonotcontactme@gmail.com} \ \tag{\faPhone}\info{+1 438-728-8285}}
%  \personalInfo{\tag{\faGithub}\info{\href{https://github.com/MGYBY}{MGYBY}} \infoSeparator\tag{Date of birth}\info{08-11-1968}}
%  \personalInfo{\faGithub \faEnvelope \faEnvelopeO \faEnvelopeSquare}
%}

\printPersonalInfo{
	\personalInfo{\tag{E-mail}\info{boyuan.yu@mail.mcgill.ca} \infoSeparator\tag{Telephone number}\info{+1 438-728-8285}		}
	\personalInfo{\tag{Website}\info{TODO}}
%	\personalInfo{\tag{Date of birth}\info{1996-03-29}}
	
}

%----------------------------------------------------------------------------------------
%   Education section
%----------------------------------------------------------------------------------------

\begin{rSection}{Education}
	
	% PhD degree entry
	\begin{rSubsectionNoBullet}{\bf PhD degree in  Civil Engineering}{McGill University}{PhD degree program}{September 2020 - May 2024}
		\italicitem{CGPA:\textit{ 4.00/4.00}.}
		\italicitem{Thesis title: TBD.}
	\end{rSubsectionNoBullet}

    % Master's degree entry
    \begin{rSubsectionNoBullet}{\bf Master's degree in  Civil Engineering}{McGill University}{Master's degree program}{September 2018 - May 2020}
        \italicitem{CGPA:\textit{ 3.96/4.00}.}
        \italicitem{Thesis title: \textit{Transverse shear instability in steep open-channel flow}, \href{https://escholarship.mcgill.ca/concern/theses/2n49t6525}{Link}.}
    \end{rSubsectionNoBullet}
    
    % Bachelor's degree entry
    \begin{rSubsectionNoBullet}{\bf Bachelor's degree in  Water Conservancy and Hydropower Engineering}{ Hohai University}{Bachelor's degree program}{September 2014 - May 2018}
        \italicitem{CGPA:\textit{4.80/5.00}.}
        \italicitem{Ranking: 1/155.}
    \end{rSubsectionNoBullet}

\end{rSection}

\begin{rSection}{Peer-Reviewed Publications}

\begin{enumerate}
	\item \textbf{Boyuan Yu}, and Vincent H. Chu,\\
	The front runner in roll waves produced by local disturbances,\\
	J. Fluid Mech., 932, A42 (2022) [18 pages];\\
	DOI:\href{https://doi.org/10.1017/jfm.2021.1011}{10.1017/jfm.2021.1011}.
	
	\item \textbf{Boyuan Yu}, and Vincent H. Chu,\\
	Impact force of roll waves against obstacles,\\
	J. Fluid Mech., \textcolor{red}{999}, A42 (2023) [\textcolor{red}{25} pages];\\
	\textcolor{red}{DOI}:  \href{https://doi.org/10.1017/jfm.2023.580}{10.1017/jfm.2023.580}.
	
	\item \textbf{Boyuan Yu}, and Vincent H. Chu,\\
	Impact Force of Roll Waves on Mudflow Modelled
	as Power-law Fluids,\\
	In prep.
	
	\item \textbf{Boyuan Yu}, and Vincent H. Chu,\\
	Wave and bed-friction effect on instability of shear flow in shallow waters,\\
	River Flow 2020 Conference, 2020 [8 pages];\\
	DOI: \href{https://doi.org/10.1201/b22619-12}{10.1201/b22619-12}.
	
	\item \textbf{Boyuan Yu}, and Vincent H. Chu,\\
	Impact force of the roll waves produced by local disturbances,\\
	the 39th IAHR World Congress, 2022 [10 pages];\\
	DOI: \href{https://doi.org/10.3850/IAHR-39WC2521711920221273}{10.3850/IAHR-39WC2521711920221273}.
	
	\item \textbf{Boyuan Yu}, and Vincent H. Chu,\\
	Roll Waves on a Laminar Sheet Flow produced by Local Disturbance,\\
	River Flow 2022 Conference, 2022 [8 pages].
	
	\item \textbf{Boyuan Yu}, and Vincent H. Chu,\\
	Impact of Mud Flow Instabilities on Hydraulic Structures,\\
	River Flow 2022 Conference, 2022 [9 pages].
	
	\item \textbf{Boyuan Yu}, and Vincent H. Chu,\\
	The Impact of Flood Waves on Hydraulic Structures,\\
	River Flow 2022 Conference, 2022 [8 pages].

\end{enumerate}

\end{rSection}

\begin{rSection}{Talks and Presentations}
\begin{enumerate}
	\item \textbf{Boyuan Yu}, and Vincent H. Chu,\\
	The video animation related to the conference paper \textit{Impact of Mud Flow Instabilities on Hydraulic Structures},\\
	River Flow 2022 Conference Best Video Contest, 2022.
\end{enumerate}
	
\end{rSection}	

%----------------------------------------------------------------------------------------
%   Work experience section
%----------------------------------------------------------------------------------------

\begin{rSection}{Teaching and Mentoring}

    % First work experience entry
    \begin{rSubsection}{Teaching Assistant}{September 2019 - May 2024}{McGill University}{Montreal, Canada}
        \item CIVE 281: Analytical Mechanics.
        \item CIVE 327: Fluid Mechanics and Hydraulics.
        \item CIVE 572: Computational Hydraulics.
    \end{rSubsection}

\end{rSection}

\begin{rSection}{Research Interests}
	\begin{itemize}
		\item Hydrodynamic instabilities.
		\item Shallow water equations.
		\item Finite volume method.
		\item Riemann solvers.
		\item Multiphase flow.
		\item Non-Newtonian fluids.
		\item Open-source CFD.
	\end{itemize}
	
\end{rSection}

%----------------------------------------------------------------------------------------
% Technical skills section
%----------------------------------------------------------------------------------------

\tableEnv{Technical skills}{
	\entry{Programming Languages/Tools}{Matlab, Mathematica, Python, C, Fortran, Julia}
    \entry{Numerical models for CFD}{Basilisk, Gerris, OpenFOAM, Clawpack, Centpy}
    \entry{Postprocessing tools for CFD}{Tecplot, Paraview, OriginLab}
    \entry{Text processing}{\LaTeX, Microsoft Word, Markdown \& Obsidian}
    \entry{Operating system}{Windows, Linux}
    \entry{Video editing}{Kdenlive}
}

%----------------------------------------------------------------------------------------
% Language proficiencies section
%----------------------------------------------------------------------------------------

%\tableEnv{Language proficiencies}{
%    \entry{English}{Fluent in Shakespearean insults}
%    \entry{French}{Proficient in baguette related jokes}
%}

\end{document}